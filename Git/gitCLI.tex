\documentclass{beamer}
\usetheme{metropolis}
%%%
\usepackage{polyglossia}
\setmainlanguage{spanish}
\setotherlanguages{english}
%%%
\usepackage{url}
\usepackage{lscape}
\usepackage[squaren]{SIunits}
\usepackage{spverbatim}
%%%
\usepackage{csquotes}
\usepackage[style=authoryear-comp, backend=bibtex]{biblatex} 
% \bibliographystyle{plainnatR} 
% backend=biber  % for biber
% apalike harvard chicago unsrtnat elsart-num-names
% \bibliography{referencias}
\addbibresource{referencias.bib}
%%%
% R/knitr
\usepackage{tikz}
\usepackage{rotating}
\usepackage{subfig}
\usepackage{inconsolata}
\usepackage{dcolumn}   % requerido por stargazer y texreg
\usepackage{booktabs}  % requerido por xtable y texreg
%%%
\usepackage{syntonly}
%\syntaxonly %verifica sintaxis sin generar documento
%%%

%%%%%%%%%%%%%%%%%% hipervinculos activos %%%%%%%%%%%%%%%%%%%%%%%

% \ifx\pdfoutput\undefined % We're not running pdftex
% \RequirePackage[bookmarks,hyperindex,plainpages=false]{hyperref}
% %colorlinks,
% \else
% \RequirePackage[bookmarks,hyperindex,plainpages=false]{hyperref}
% \def\pdfBorderAttrs{/Border [0 0 0] } % No border arround Links
% \hypersetup{pdfauthor={Ulises M. Alvarez}}
% \hypersetup{pdftitle={Sophie UNAM}}
% \fi
%%%%%%%%%%%%%%%%%%

\title{Git II - Desde la l\'\i{}nea de comandos}
% Use metropolis theme
\date{\today}

\author{Ulises M. Alvarez\\ %
%   Proyecto Sophie UNAM\\ %
   $<$\href{mailto:uma@sophie.unam.mx}% 
   {uma@sophie.unam.mx}$>$
}

\institute{Sophie UNAM}

\begin{document}

\maketitle

% \begin{frame}{Brevario}
%   \setbeamertemplate{section in toc}[sections numbered]
%   \tableofcontents[hideallsubsections]
% \end{frame}

% \section{Caracter\'isticas a considerar}
% \label{sec:caracteristicas}

\begin{frame}[standout]
  Resumen
\end{frame}

\begin{frame}
  \frametitle{Git 101}
  \begin{itemize}
  \item Es un software de control de versiones diseñado por Linus
    Torvalds (2005).
  \item Lleva registro de cambios en archivos.
  \item Puede funcionar a nivel \textit{directorio}, no requiere
    acceso a Internet ni de un servidor central.
  \item Presente en los principales sistemas operativos.
  \end{itemize}
\end{frame}

\begin{frame}[standout]
  C\'omo funciona
\end{frame}

\begin{frame}[fragile]
  \frametitle{DIFF(1)}
\begin{verbatim}
DIFF(1) 

NAME
       diff - compare files line by line

SYNOPSIS
       diff [OPTION]... FILES

DESCRIPTION
       Compare FILES line by line.

       -u, -U NUM, --unified[=NUM]
              output NUM (default 3) lines of unified context
\end{verbatim}  
\end{frame}

\begin{frame}[fragile]
  \frametitle{\texttt{diff} en acci\'on}
\begin{verbatim}
$ pwd
/home/uma/ownCloud/TeX/github/examples
$ ls
httpd.conf      httpd.conf.~2~  httpd.conf.~4~  my.cnf.~1~   php-7.0.ini.~1~  php-fpm.conf.~1~
httpd.conf.~1~  httpd.conf.~3~  my.cnf          php-7.0.ini  php-fpm.conf     php-fpm.conf.~2~
\end{verbatim}
\end{frame}

\begin{frame}[fragile]
  \frametitle{\texttt{diff} en acci\'on}
\begin{verbatim}
$ for i in httpd.conf*; do wc -l $i; done
      38 httpd.conf
      28 httpd.conf.~1~
      28 httpd.conf.~2~
      38 httpd.conf.~3~
      38 httpd.conf.~4~
\end{verbatim}
\end{frame}

\begin{frame}[fragile]
  \frametitle{\texttt{diff} en acci\'on}
\begin{verbatim}
$ diff -p httpd.conf.~1~ httpd.conf.~2~
5c5
< server "nextcloud.example.com" {
---
> server "janikua.geociencias.unam.mx" {
7c7
<       listen on $ext_if tls port 443
---
>       #listen on $ext_if tls port 443
\end{verbatim}
\end{frame}

\begin{frame}[fragile]
  \frametitle{\texttt{diff} en acci\'on}
\begin{verbatim}
$ diff -p httpd.conf.~1~ httpd.conf.~2~ >httpd-01.patch
$ wc -l httpd-01.patch
       8 httpd-01.patch
\end{verbatim}
\end{frame}


\begin{frame}[fragile]
  \frametitle{PATCH(1)}
\begin{verbatim}
PATCH(1)

NAME
     patch - apply a diff file to an original

     -C, --check
             Checks that the patch would apply cleanly, but does not modify
             anything.

     -p strip-count, --strip strip-count
             Setting -p0 gives the entire pathname unmodified. 
\end{verbatim}  
\end{frame}

\begin{frame}[fragile]
  \frametitle{\texttt{patch} en acci\'on}
\begin{verbatim}
$ mkdir tmp
$ cp {httpd.conf.~1~,httpd-01.patch} tmp/
$ cd tmp/
$ sha1 *
SHA1 (httpd-01.patch) = 6ff58bedf80de760a2d1694636bd443a858cbe24
SHA1 (httpd.conf.~1~) = 5f5eb1b845e9e611be85f51baf06a40317c0d287
\end{verbatim}
\end{frame}

\begin{frame}[fragile]
  \frametitle{\texttt{patch} en acci\'on}
\begin{verbatim}
$ patch -C -p0 httpd.conf.~1~ httpd-01.patch  
Hmm...  Looks like a normal diff to me...
Patching file httpd.conf.~1~ using Plan A...
Hunk #1 succeeded at 5.
Hunk #2 succeeded at 7.
\end{verbatim}
\end{frame}

\begin{frame}[fragile]
  \frametitle{\texttt{patch} en acci\'on}
\begin{verbatim}
$ patch -p0 httpd.conf.~1~ httpd-01.patch
Hmm...  Looks like a normal diff to me...
Patching file httpd.conf.~1~ using Plan A...
Hunk #1 succeeded at 5.
Hunk #2 succeeded at 7.
done
$ sha1 *                                  
SHA1 (httpd-01.patch) = 6ff58bedf80de760a2d1694636bd443a858cbe24
SHA1 (httpd.conf.~1~) = be5b367ecb497351a83a71b664500f383503b8c3
SHA1 (httpd.conf.~1~.orig) = 5f5eb1b845e9e611be85f51baf06a40317c0d287
\end{verbatim}
\end{frame}

\begin{frame}[fragile]
  \frametitle{\texttt{patch} revirtiendo cambios}
\begin{verbatim}
$ patch -C -p0 httpd.conf.~1~ httpd-01.patch        
Hmm...  Looks like a normal diff to me...
Patching file httpd.conf.~1~ using Plan A...
Reversed (or previously applied) patch detected!  Assume -R? [y] 
Hunk #1 succeeded at 5.
Hunk #2 succeeded at 7.
done
\end{verbatim}
\end{frame}

\begin{frame}[standout]
  \texttt{GIT} desde la l\'\i{}nea de comandos
\end{frame}


\begin{frame}[fragile]
  \frametitle{\texttt{GIT} en acci\'on}
\begin{verbatim}
$ pwd
/home/uma/ownCloud/TeX/github/examples/tmp
$ cd ../
$ mkdir my1stgit
$ cd my1stgit/
$ git config --global user.name "Ulises M. Alvarez"
$ git config --global user.email uma@sophie.unam.mx
\end{verbatim}
\end{frame}


\begin{frame}[fragile]
  \frametitle{Importando un proyecto}
  Para importar un nuevo proyecto y ponerlo bajo GIT:
\begin{verbatim}
$ cp ../httpd.conf.~1~ httpd.conf
$ git init
Initialized empty Git repository in .../my1stgit/.git/
git add .
\end{verbatim}
\end{frame}

\begin{frame}[fragile]
  \frametitle{Almacenando: \texttt{git commit}}
\begin{verbatim}
$ git commit
Este es mi primer repo de GIT.
#
# Initial commit
#
# Changes to be committed:
#       new file:   httpd.conf
#
~
~
~
[master (root-commit) 7ca8a8e] Este es mi primer repo de GIT.
 1 file changed, 28 insertions(+)
 create mode 100644 httpd.conf
\end{verbatim}
\end{frame}

\begin{frame}[fragile]
  \frametitle{Haciendo cambios}
  Editamos el archivo y agregamos los cambios\ldots{}
\begin{verbatim}
$ emacs httpd.conf
$ git add httpd.conf
\end{verbatim}
\end{frame}

\begin{frame}[fragile]
  \frametitle{Haciendo cambios}
  Para ver el estado general:
\begin{verbatim}
$ git status        
On branch master
Changes to be committed:
  (use "git reset HEAD <file>..." to unstage)

        modified:   httpd.conf
\end{verbatim}
\end{frame}

\begin{frame}[fragile]
  \frametitle{Registrando el cambio}
\begin{verbatim}
$ git commit
Se modifica el nombre del servidor y se bloquea el puerto 80.
# On branch master
# Changes to be committed:
#       modified:   httpd.conf
#
~

[master be1ec04] Se modifica el nombre del servidor y se bloquea el puerto 80.
 1 file changed, 2 insertions(+), 2 deletions(-)
\end{verbatim}
\end{frame}

\begin{frame}[fragile]
  \frametitle{Verificando}
\begin{verbatim}
$ git status 
On branch master
nothing to commit, working tree clean

\end{verbatim}
\end{frame}

% \begin{frame}{Lenguajes: Popularidad}
%   \begin{figure}[hp]
%     \centering \includegraphics[width=0.75\textwidth]{IEEE02}
%     \caption{\href{http://spectrum.ieee.org/static/interactive-the-top-programming-languages-2016}%
%     {Tendencias en IEEE (II).}  }
%     \label{fig:ieee02}
%   \end{figure}
% \end{frame}

% % \section{Conclusiones}
% % \label{sec:conlusiones}


% \begin{frame}[standout]
%   Qué lenguaje usar
% \end{frame}

% \begin{frame}{Conclusiones}
%   \begin{itemize}
%   \item Lo que use mi gremio (\emph{siempre y cuando sea de código
%     abierto}).
%   \item El que se integre mejor a mi flujo de trabajo.
%   \item Perspectiva de desarrollo académico (\emph{no siempre el más
%     usado es el mejor}).
%   \end{itemize}
% \end{frame}


\end{document}
%%% Local Variables:
%%% mode: latex
%%% mode: flyspell
%%% TeX-engine: luatex
%%% End:
